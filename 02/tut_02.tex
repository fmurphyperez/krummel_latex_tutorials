\documentclass{article}
%\usepackage{unicode-math}
%\newcommand\Alpha{\mathrm{A}}
\newcommand{\Alpha}{\mathrm{A}}
\begin{document}
Superscripts:
\[x^1\] % lazy mode!
\[x^12\] % lazy mode with more than one digit will be wrong!
\[x^{12}\] % use curly braces!
%\[x^1^2\] % Nested subscripts won't work in lazy mode
\[x^{1^2}\] % This will work, which is the same as...
\[x^{1^{2}}\] %
\[x^{1^{2^{3}}}\] % now trying more than two

Subscripts:
\[x_1\] % lazy mode!
\[x_12\] % lazy mode with more than one digit will be wrong!
\[x_{12}\] % use curly braces!
%\[x_1_2\] % Nested subscripts won't work in lazy mode
\[x_{1_2}\] % This will work, which is the same as...
\[x_{1_{2}}\] %
\[x_{1_{2_{3}}}\] % now trying more than two 

% Conclusión, no sea floj@. 

Greek letters:
\[\pi\]
\[\Pi\]
\[\alpha\] % easy-peasy
\[\Alpha\] % not so easy-peasy 
% If xelatex, do \usepackage{unicode-math} in the preamble!
% If pdflatex, do \newcommand{\Alpha}{\mathrm{A}}
%\[A=\pir^{2}\] % This won't work.
\[A=\pi r^{2}\] % This will work.
\[A=\pi{}r^{2}\] % This will also work.

% Conclusión, use la versión que más le guste.
Trig functions:
\[y=sin{}x\] % This is wrong.
\[y=\sin{x}\] % This is how it should be.
\[y=\cos{x}\]
\[y=\cos{\theta}\] % Makes sense, why we use braces.
% Nueva conclusión, no use la versión que más le guste, use {}.
\[y=\sin^{-1}{\theta}\]
\[y=\arcsin{\theta}\] % Also works like this.

Log functions:
\[y=logx\] % Wrong!
\[y=\log{x}\]
\[y=\log_{12}{x}\]
\[y=\ln{x}\]

Roots:
\[\sqrt{3}\]
\[\sqrt[3]{2}\] % easy-peasy
% Turn off the page numbers? \pagestyle{empty}
\[\sqrt{x^2+y^2}\]
\[\sqrt{3\sqrt{2}}\]

Fractions:
\[\frac{2}{3}\]
% In a line
About \(\frac{2}{3}\) of the glass is full, I guess.\\
%
About \(\frac{2}{3}\) of the glass is full, I guess.\\ [2pt] % ugly!
%
About dos terceras partes of the glass is full, I guess. % Do you really need a fraction in a text line?
% Try something more difficult
\[\frac{1}{\sqrt{\frac{1}{x}+1}}\]



\end{document}