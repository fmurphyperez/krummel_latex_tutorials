\documentclass{article}
\usepackage{amsfonts} % to use \mathbb{text}
%\usepackage{float} % to use HERE option in table environment or just use h
\usepackage{amsmath} % environment align
\parindent 0pt
\begin{document}
\section{Special Symbols}
  The distributive property states that \(a(b+c)=ab+ac\), for all \(a, b, c \in \mathbb{R}\).
  
  The equivalence class of \(a\) is \([a]\). 
  The set \(A\) is defined to be \({1, 2, 3}\). % So, curly brackets should be escaped to use them.
  %The set \(A\) is defined to be \(\{1, 2, 3\}\). 
  % It happens the same with dollar sign.
  The movie ticket costs \$10 USD.
  
\section{Parentheses}
  \[2(\frac{1}{x^{2}-1})\] % The problem with this is that the parentheses are a little short.
  \[2\left(\frac{1}{x^{2}-1}\right)\]
  % Works with lots of enclosing symbols (), {}, [], <>, ||, arrows.
  \[2\left\Uparrow\frac{1}{x^{2}-1}\right\Uparrow\]
  % What if I want only one symbol?
  \[\frac{dy}{dx}|_{x=1}\] % looks bad..
  \[\left.\frac{dy}{dx}\right|_{x=1}\]
  % Something more difficult
  \[\left(\frac{1}{1+\left(\frac{1}{1+x}\right)}\right)\]
  
\section{Tables}
  \begin{tabular}{|c|c|c|c|c|c|}
    \hline
    \(x\) & 2 & 4 & 6 & 8 & 10 \\ \hline
    \(f(x)\) & 10 & 11 & 12 & 13 & 14 \\
    \hline
  \end{tabular}
\vspace{1cm}
\begin{table}[h]
  \centering
  \def\arraystretch{1.5}
  \begin{tabular}{|c|c|c|c|c|c|}
    \hline
    \(x\) & 2 & 4 & 6 & 8 & 10 \\ \hline
    \(f(x)\) & $\frac{1}{2}$ & 11 & 12 & 13 & 14 \\
    \hline
  \end{tabular}
  \caption{Something.}
\end{table} % Frustrating... floats move!
\begin{table}[h]
  \centering
  \def\arraystretch{1.5}
  \begin{tabular}{|c|c|}
    \hline
    \(f(x)\) & \(f'(x)\) \\ \hline
    \(x>0\) & The function \(f(x)\) is increasing. \\
    \hline
  \end{tabular}
  \caption{Something else.}
\end{table}
% Tables are numbered.
\begin{table}[h]
  \centering
  \def\arraystretch{1.5}
  \begin{tabular}{|c|p{5.5cm}|} % from |c|c|!!
    \hline
    \(f(x)\) & \(f'(x)\) \\ \hline
    \(x>0\) & The function \(f(x)\) is increasing and we have a lot of text here. Damn it! It does not fit in the page! \\
    \hline
  \end{tabular}
  \caption{Something else.}
\end{table}
\section{Equation Arrays}
  \begin{align}
    5x^{2}-9=x+3\\
    5x^{2}-x-12=0
  \end{align}
      \begin{align*} % see the difference
    5x^{2}-9&=x+3\\
    5x^{2}-x-12&=0 \\
    &=12+x-5^x{2}
  \end{align*}
    \begin{align} % see the difference
    5x^{2}-9&=x+3\\
    5x^{2}-x-12&=0 \\
    &=12+x-5^x{2}
  \end{align}
\end{document}